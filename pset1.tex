%Jennifer Pan, August 2011

\documentclass[10pt,letter]{article}
	% basic article document class
	% use percent signs to make comments to yourself -- they will not show up.

\usepackage{amsmath}
\usepackage{amssymb}
	% packages that allow mathematical formatting

\usepackage{graphicx}
	% package that allows you to include graphics

\usepackage{setspace}
	% package that allows you to change spacing

\onehalfspacing
	% text become 1.5 spaced

\usepackage{fullpage}
	% package that specifies normal margins
	

\begin{document}
	% line of code telling latex that your document is beginning


\title{Problem Set}

\author{Gregory Nisbet}

\date{Due Date, 2010}
	% Note: when you omit this command, the current date is automatically included
 
\maketitle 
	% tells latex to follow your header (e.g., title, author) commands.

\section*{Definitions and Notation}

A collection is just a plural individual. It is not a set and is not a sequence.

Let $a|b$ mean that $a$ divides $b$.

Let $a \le_D b$ be an alternative notation for $a|b$.

As a convention, instead of writing $x=y=z$, I will write $1 = \{x, y, z\}$.


Let $R$ be a commutative ring.

Let $a$ and $b$ be elements of $R$.

Let $\text{comdiv}(a, b)$ be a collection denoting the common divisors of $a$ and $b$.

$$ d \in \text{comdiv}(a, b) \stackrel{\text{def}}{\iff} d | a \land d | b $$

Using this, we can define the gcd. Note that the gcd need not be unique.

$$ d = \text{gcd}(a, b) \stackrel{\text{def}}{\iff} d \in \text{comdiv}(a, b) \;\;\text{and}\;\; \forall w \in \text{comdiv}(a, b) \text{.} w | d $$

More generally, the following holds:

$$ \text{comdiv}(a, b) \le_D \text{gcd}(a, b) $$

The gcd fails to exist if and only if $\text{comdiv}(a, b)$ with divisibility as $\le$ does not have a maximal element.

The gcd exists but fails to be unique if and only if there are multiple maximal elements.

The lcm is defined analogously, but

$$ \text{lcm}(a, b) \le_D \text{commul}(a, b) $$

\section*{Problem 11.2.12a}

Let $R$ be a principal ideal domain. Prove that there is a least common multiple $[a,m]=b$ of two elements which are not
both zero such that $a$ and $b$ divide $m$, and that if $a, b$ divide an element $r \in R$, then $m$ divides $r$. Prove that $m$ is unique
up to unit factor.

\paragraph{A)} Answer to 11.2.12a

Referring back to my definitions, the set of common multiples of $\{a, b\}$ (written $\text{commul}(a,b)$) always exists.

\subparagraph{i)} $\text{commul}(a, b)$ is not empty.

Let $\langle a \rangle$ refer to the ideal in $\mathbb{R}$ generated by $a$. The common multiples of $a$ and $b$ are
precisely the elements that are in both $\langle a \rangle$ and $\langle b \rangle$ .

$$ \text{commul}(a, b) = \langle a \rangle \cap \langle b \rangle $$

$ab$ is in $\text{commul}(a, b)$, therefore it is not empty.

\subparagraph{ii)} $\text{commul}(a, b)$ has at least one least element.

$\text{commul}(a, b)$ is the intersection of two ideals $\langle a \rangle$ and $\langle b \rangle$.

The intersection of two ideals is an ideal.

However, by hypothesis, $R$ is a principal ideal domain. Therefore $\langle a \rangle \cap \langle b \rangle$ has a single generator if it is a proper ideal.

Suppose $\langle a \rangle \cap \langle b \rangle$ is not a proper ideal, then it is the whole ring $R$. This means that $1 = \{ \langle a \rangle, \langle b \rangle, R \}$.
Therefore $R$ is generated by $a$.

Therefore $\langle a \rangle \cap \langle b \rangle$ is always a principal ideal. Let's call the generator $k$.

In $I = \langle k \rangle$, $k$ divides every element of $I$.

Therefore $k$ is an lcm.

\subparagraph{iii)} If $\text{commul}(a, b)$ has multiple least elements, then they are associates.

Let $k$ be a generator of $I = \langle a \rangle cap \langle b \rangle$.

Suppose $I$ is also generated by $j \ne k$.

This means that $\{k, j\} \subset I $. This means that $j \le_D k$ and that $k \le_D j$.

This means that there exists some factor $a$ such that $j = ak$.

However, the fact that $k \le_D j$ means that $a$ must be invertible.

If $a$ is invertible, then by definition it is a unit.

Therefore $j$ and $k$ are associates. 

\section{*Problem 11.2.12b}

Let $R$ be a principal ideal domain. Denote the greatest common divisor of $a$ and $b$ by $(a, b)$. Prove that $(a, b)[a,b]$ is an associate of $ab$.

\paragraph{A)} Answer to 11.2.12b

We want to show the following:

$$ \text{gcd}(a,b)\text{lcm}(a,b) \;\;\text{is an associate of}\;\; ab $$

The $\text{gcd}(a,b)$ is an arbitrary maximal element of $\text{comdiv}(a, b)$.

I dunno.

 
\section*{Problem 11.3.2}

\section*{Problem 11.3.3}

\section*{Problem 11.3.4}

\section*{Problem 11.3.8}

\section*{Problem 11.3.9}

\section*{Problem 11.3.10}


\end{document}
	% line of code telling latex that your document is ending. If you leave this out, you'll get an error
